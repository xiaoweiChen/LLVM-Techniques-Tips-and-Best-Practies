In this section, we learned about LLVM IR – the target-independent intermediate representation that sits at the core of the entire LLVM framework. We provided an introduction to the high-level structure of LLVM IR, followed by practical guidelines on how to walk through different units within its hierarchy. We also focused on instructions, values, and SSA form at, which are crucial for working with LLVM IR efficiently. We also presented several practical skills, tips, and examples on the same topic. Last but not least, we learned how to process loops in LLVM IR – an important technique for optimizing performance-sensitive applications. With these abilities, you can perform a wider range of program analysis and code optimization tasks on LLVM IR.

In the next chapter, we will learn about a collection of LLVM utilities APIs that can improve your productivity when it comes to developing, diagnosing, and debugging with LLVM.