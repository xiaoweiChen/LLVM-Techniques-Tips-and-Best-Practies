
现代编译器优化可能很复杂,通常需要来自目标程序的大量信息,以便做出正确的决策和最佳的转换,例如:在为新PassManager编写一个LLVM Pass时,LLVM会使用\texttt{noalias}属性来计算内存别名信息,这可以用来删除冗余的内存负载。

其中一些信息——在LLVM中称为\textbf{分析}——评估代价相当昂贵。此外,单个分析也可能依赖于其他分析。因此,LLVM创建\texttt{AnalysisManager}组件来处理所有与LLVM分析相关的任务。本节中,我们将展示如何在自己的Pass中使用AnalysisManager,以便编写更强大和更复杂的程序转换或分析。我们还将使用一个示例项目\textbf{HaltAnalyzer}来演示我们的教程。下一节将提供HaltAnalyzer的概述,然后再进入详细的开发阶段。

\subsubsubsection{9.3.1\hspace{0.2cm}项目概述}

HaltAnalyzer是在一个场景中设置的,目标程序使用一个特殊的函数\texttt{my\_halt},当它调用时终止程序的执行。\texttt{my\_halt}函数类似于\texttt{std::terminate}函数,或者\texttt{assert}函数(在完整性检查失败时)。

HaltAnalyzer的工作是分析程序,以找到因\texttt{my\_halt}函数而不可能运行的程序块。更具体地说,以如下代码为例:

\begin{lstlisting}[style=styleCXX]
int foo(int x, int y) {
	if (x < 43) {
		my_halt();
		if (y > 45)
		return x + 1;
		else {
			bar();
			return x;
		}
	} else {
		return y;
	}
}
\end{lstlisting}

因为\texttt{my\_halt}在\texttt{if (x < 43)}语句的true分支的开头调用,所以该分支之后的代码永远不会执行(也就是说,在到达这些行之前,\texttt{my\_halt}就停止了所有的程序执行)。

HaltAnalyzer应该识别这些基本块,并将警告消息输出到\texttt{stderr}。就像上一节的示例项目一样,HaltAnalyzer也是一个封装在插件中的Pass函数。因此,如果使用前面的代码作为HaltAnalyzer Pass的输入,应该打印出如下信息:

\begin{tcblisting}{commandshell={}}
$ opt --enable-new-pm --load-pass-plugin ./HaltAnalyzer.so \
      --disable-output ./test.ll
[WARNING] Unreachable BB: label %if.else
[WARNING] Unreachable BB: label %if.then2
$
\end{tcblisting}

\texttt{\%if.elese}和\texttt{\%if.then2}字符串只是\texttt{if (y > 45)}语句中基本块的名称(不同的设备可能会看到不同的名称)。另一件值得注意的事情是\texttt{-\,-disable-output}命令行标志。默认情况下,\texttt{opt}程序无论如何都会以二进制形式(即LLVM位码)输出LLVM IR,除非用户通过\texttt{-o}标志将输出重定向到其他地方。使用上述标志只是告诉\texttt{opt}不要这样做,因为我们对LLVM IR的最终内容不感兴趣(不会去修改它)。

虽然HaltAnalyzer的算法看起来非常简单,但从头编写可能会很痛苦。这里,我们利用LLVM提供的分析:\textbf{支配树(Dominator Tree, DT)}。\textbf{控制流图(CFG)}的概念已经在大多数入门级编译器课程中,所以我们不打算在这里深入解释它。简单地说,如果我们说一个基本块支配另一个块,每一个到达后者的执行流都保证先经过前者。DT是LLVM中最重要、最常用的分析方法之一,大多数与控制流相关的转换离不开它。

把这个想法放到HaltAnalyzer中,我们只是简单地寻找所有的基本块,这些基本块包含对\texttt{my\_halt}的函数调用(我们从警告消息中排除了包含\texttt{my\_halt}调用的基本块)。下一节中,我们将详细说明如何编写HaltAnalyzer。


\subsubsubsection{9.3.2\hspace{0.2cm}编写HaltAnalyzer Pass}

这个项目中,我们只创建一个源文件\texttt{HaltAnalyzer.cpp}。大多数基础设施,包括\texttt{CMakeListst.txt},都可以从上一节的\texttt{StrictOpt}项目中拿过来直接用:

\begin{enumerate}
\item 在\texttt{HaltAnalyzer.cpp}中,我们首先创建以下Pass框架:

\begin{lstlisting}[style=styleCXX]
class HaltAnalyzer : public PassInfoMixin<HaltAnalyzer> {
	static constexpr const char* HaltFuncName = "my_halt";
	// All the call sites to "my_halt"
	SmallVector<Instruction*, 2> Calls;
	void findHaltCalls(Function &F);
public:
	PreservedAnalyses run(Function &F,
	FunctionAnalysisManager &FAM);
};
\end{lstlisting}

除了前一节中看到的\texttt{run}方法之外,我们还创建了一个额外的方法\texttt{findHaltCalls},它将收集当前函数中所有对\texttt{my\_halt}的指令调用,并将它们存储在\texttt{calls} vector中。

\item 先实现\texttt{findHaltCalls}:

\begin{lstlisting}[style=styleCXX]
void HaltAnalyzer::findHaltCalls(Function &F) {
	Calls.clear();
	for (auto &I : instructions(F)) {
		if (auto *CI = dyn_cast<CallInst>(&I)) {
			if (CI->getCalledFunction()->getName() ==
			HaltFuncName)
			Calls.push_back(&I);
		}
	}
}
\end{lstlisting}

这个方法使用\texttt{llvm::instructions}来遍历当前函数中的每个\texttt{Instruction}调用,并逐个检查。如果指令调用是\texttt{CallInst}——代表典型的函数调用点——并且调用函数的名字是\texttt{my\_halt},我们将把它放入到\texttt{Calls} vector中以供以后使用。

\begin{tcolorbox}[colback=blue!5!white,colframe=blue!75!black, fonttitle=\bfseries,title=重建函数名]	
\hspace*{0.7cm}请注意,当一行C++代码编译成LLVM IR或本机代码时,任何符号的名称——包括函数名——都将与原始源代码中的不同,例如:一个简单的函数,名称\texttt{foo},没有参数,可能在LLVM IR中称为\texttt{\_Z3foov}。在C++中,我们称这种转换为\textbf{名称重建}。不同的平台也会采用不同的名称转换方案,例如:Visual Studio中,相同的函数名在LLVM IR中变成\texttt{?foo@@YAHH@Z}。
\end{tcolorbox}

\item 现在,让我们回到\texttt{HaltAnalyzer::run}方法。要做两件事,将通过\texttt{findHaltCalls}收集\texttt{my\_halt}的调用点(这是我们刚刚写的),然后检索DT,分析数据:

\begin{lstlisting}[style=styleCXX]
#include "llvm/IR/Dominators.h"
…
PreservedAnalyses
HaltAnalyzer::run(Function &F, FunctionAnalysisManager
&FAM) {
	findHaltCalls(F);
	DominatorTree &DT = FAM.
	getResult<DominatorTreeAnalysis>(F);
	…
}
\end{lstlisting}

前面代码中,向我们展示了如何利用\texttt{FunctionAnalysisManager}类型参数来检索特定\texttt{Function}类,再进行特定的分析数据(在本例中为\texttt{DominatorTree})。

目前为止,我们(在某种程度上)交替使用了分析和分析数据这两个词,但在实际的LLVM实现中,它们是两个不同的实体。以我们这里用的DT为例:

\begin{enumerate}[label=\alph*)]
\item \textbf{DominatorTreeAnalysis}是一个C++类,计算给定\texttt{Function}的关系。换句话说,它是执行分析的对象。

\item \textbf{DominatorTree}是一个C++类,表示\texttt{DominatorTreeAnalysis}生成的结果。这只是静态数据,AnalysisManager会缓存它直到它失效为止。

\end{enumerate}

此外,LLVM要求每个分析通过\texttt{Result}成员类型来搞清其附属的结果类型。例如:\texttt{DominatorTreeAnalysis::Result}等于\texttt{DominatorTree}。

为了使其更加正式,为了将分析类\texttt{T}的分析数据与\texttt{Function}变量\texttt{F}关联起来,可以使用如下的代码:

\begin{lstlisting}[style=styleCXX]
// `FAM` is a FunctionAnalysisManager
typename T::Result &Data = FAM.getResult<T>(F);
\end{lstlisting}

\item 在检索\texttt{DominatorTree}之后,是时候找到之前收集到的\texttt{Instruction}调用点,并控制的所有基本块了:

\begin{lstlisting}[style=styleCXX]
PreservedAnalyses
HaltAnalyzer::run(Function &F, FunctionAnalysisManager
&FAM) {
	…
	SmallVector<BasicBlock*, 4> DomBBs;
	for (auto *I : Calls) {
		auto *BB = I->getParent();
		DomBBs.clear();
		DT.getDescendants(BB, DomBBs);
		for (auto *DomBB : DomBBs) {
			// excluding the block containing `my_halt` call site
			if (DomBB != BB) {
				DomBB->printAsOperand(
				errs() << "[WARNING] Unreachable BB: ");
				errs() << "\n";
			}
		}
	}
	return PreservedAnalyses::all();
}
\end{lstlisting}

通过使用\texttt{DominatorTree::getDescendants}方法,我们可以检索\texttt{my\_halt}调用点所控制的所有基本块。注意,\texttt{getDescendants}的结果也将包含放入的查询块(本例中,是包含\texttt{my\_halt}调用点的块),因此需要我们在使用\texttt{BasicBlock::printAsOperand}方法打印基本块名称之前将其排除。

返回\texttt{PreservedAnalyses::all()}的末尾,它告诉AnalysisManager这个Pass不会使任何分析失效(因为没有修改IR),我们将在这里包装\texttt{HaltAnalyzer::run}方法。

\item 最后,我们需要动态地将HaltAnalyzer Pass插入到Pass流水中。使用与上一节相同的方法,通过实现\texttt{llvmGetPassPluginInfo}函数,并使用\texttt{PassBuilder}将我们的Pass放在流水中的某些EP上:

\begin{lstlisting}[style=styleCXX]
extern "C" ::llvm::PassPluginLibraryInfo LLVM_ATTRIBUTE_
WEAK
llvmGetPassPluginInfo() {
	return {
		LLVM_PLUGIN_API_VERSION, "HaltAnalyzer", "v0.1",
		[](PassBuilder &PB) {
			using OptimizationLevel
			= typename PassBuilder::OptimizationLevel;
			PB.registerOptimizerLastEPCallback(
			[](ModulePassManager &MPM, OptimizationLevel OL)
			{
				MPM.addPass(createModuleToFunctionPassAdaptor
				(HaltAnalyzer()));
			});
		}
	};
}
\end{lstlisting}

与前一节中的\texttt{StrictOpt}相比,我们使用\texttt{registeroptimizerlastcallback}在所有其他优化Pass之后插入HaltAnalyzer。这背后的基本原理是,一些优化可能会修改基本块,所以过早地提示警告可能不是很有用。尽管如此,我们仍然利用\texttt{ModuletoFunctionPassAdaptor}来包装我们的Pass,这是因为\texttt{registeroptimizerlastcallback}只提供了\texttt{ModulePassManager},让我们添加Pass(其实就是一个Pass函数)。

\end{enumerate}

这些就是实现HaltAnalyzer的所有步骤了。现在,我们已经了解了如何使用LLVM的程序分析基础结构,来获取LLVM Pass中关于目标程序的更多信息。在开发一个Pass时,这些技能可以让开发者对IR有更深入的理解。此外,这个设施允许重用来自LLVM的高质量、现成的程序分析算法,而不是自己重新创建轮子。要浏览LLVM提供的所有可用分析,在源代码树中的\texttt{llvm/include/llvm/Analysis}文件夹中的大多数头文件,都是可以使用的独立分析数据文件。

本章的最后一节中,我们将展示一些诊断技术,这些技术对调试LLVM Pass非常有用。























