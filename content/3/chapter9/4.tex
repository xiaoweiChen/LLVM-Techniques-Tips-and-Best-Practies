
LLVM中的PassManager和AnalysisManager挺复杂,它们管理数百个Pass和分析之间的交互,当我们试图诊断由它们引起的问题时,可能会是一个严峻的挑战。此外,编译器工程师通常会修复编译器中崩溃或\textbf{编译错误}的问题。在这些场景中,提供了对Pass和Pass流水有用的设施可以极大地提高修复这些问题的效率。幸运的是,LLVM已经想到了这点,并提供了许多这样的工具。

\begin{tcolorbox}[colback=blue!5!white,colframe=blue!75!black, fonttitle=\bfseries,title=编译错误]	
\hspace*{0.7cm}\textit{编译错误}通常指编译程序中的逻辑问题,这是由编译器引入,例如:过于激进的编译器优化会删除某些不应该删除的循环,从而导致编译后的软件故障,或者错误地重新排序内存屏障,从而使生成的代码中出现竞争条件。
\end{tcolorbox}

在下面的每个部分中,我们将每次介绍一个工具。以下是他们的列表:

\begin{itemize}
\item 打印通道流水细节
\item 打印每次Pass对IR的修改
\item 将Pass流水一分为二
\end{itemize}

这些工具可以在\texttt{opt}的命令行界面中进行交互。事实上,你也可以创建自己的设施(甚至不需要改变LLVM源树!),我们会将把这个留给读者们作为练习。

\subsubsubsection{9.4.1\hspace{0.2cm}打印通道流水细节}

LLVM中有许多不同的优化级别,即我们在使用\texttt{clang}(或\texttt{opt})时熟悉的\texttt{-O1}、\texttt{-O2}或\texttt{-Oz}标志。每个优化级别运行一组不同的Pass,并将它们按照不同的顺序排列。在某些情况下,这可能会在性能或正确性方面极大地影响生成的代码。有时候,为了清楚地理解我们将要处理的问题,了解这些配置很重要。

要打印出所有的Pass,以及它们当前在\texttt{opt}中运行的顺序,我们可以使用\texttt{-\,-debug-pass-manager}标志,例如:给定下面的C代码\texttt{test.c},我们将看到以下内容:

\begin{lstlisting}[style=styleCXX]
int bar(int x) {
	int y = x;
	return y * 4;
}
int foo(int z) {
	return z + z * 2;
}
\end{lstlisting}

首先使用以下命令生成IR:

\begin{tcblisting}{commandshell={}}
$ clang -O0 -Xclang -disable-O0-optnone -emit-llvm -S test.c
\end{tcblisting}

\begin{tcolorbox}[colback=blue!5!white,colframe=blue!75!black, fonttitle=\bfseries,title=-disable-O0-optnone标志]	
\hspace*{0.7cm}默认情况下,\texttt{clang}将为\texttt{-O0}优化级别下的每个函数附加一个特殊属性\texttt{optnone}。此属性将阻止附加函数的任何进一步优化。这里,\texttt{-disableo-optnone} (前端)标志会阻止\texttt{clang}附加这个属性。
\end{tcolorbox}

然后,使用以下命令打印出所有在\texttt{-O2}优化级别下运行的Pass:

\begin{tcblisting}{commandshell={}}
$ opt -O2 --disable-output --debug-pass-manager test.ll
Starting llvm::Module pass manager run.
…
Running pass: Annotation2MetadataPass on ./test.ll
Running pass: ForceFunctionAttrsPass on ./test.ll
…
\end{tcblisting}

\begin{tcblisting}{commandshell={}}
Starting llvm::Function pass manager run.
Running pass: SimplifyCFGPass on bar
Running pass: SROA on bar
Running analysis: DominatorTreeAnalysis on bar
Running pass: EarlyCSEPass on bar
…
Finished llvm::Function pass manager run.
…
Starting llvm::Function pass manager run.
Running pass: SimplifyCFGPass on foo
…
Finished llvm::Function pass manager run.
Invalidating analysis: VerifierAnalysis on ./test.ll
… $
\end{tcblisting}

前面的命令行输出告诉我们,\texttt{opt}首先运行一组模块级优化,这些Pass的顺序(例如,\texttt{Annotation\\2MetadataPass}和\texttt{ForceFunctionAttrsPass})也进行列出。之后,在对\texttt{foo}函数运行这些优化之前,对\texttt{bar}函数(例如,\texttt{SROA})执行一系列函数级优化。此外,还显示了流水中使用的分析(例如,\texttt{DominatorTreeAnalysis}),以及一条提示信息,说明已失效(由某个Pass)组件。

总而言之,\texttt{-\,-debug-pass-manager}是一个有用的工具,可以查看Pass以及在某个优化级别上,Pass流水运行的顺序。了解这些信息可以了解Pass和分析如何与输入IR进行交互。

\subsubsubsection{9.4.2\hspace{0.2cm}打印每次Pass对IR的修改}

要了解特定转换Pass对目标程序的影响,最直接的方法是比较该Pass处理IR之前和之后的IR。更具体地说,我们对特定转换Pass所做的更改感兴趣,例如:LLVM错误地删除了一个它不应该进行的循环,我们想知道Pass会做什么,以及什么时候在Pass流水中进行了删除操作。

通过在\texttt{opt}中使用\texttt{-\,-print-changed}标志(以及稍后介绍的其他一些标志),我们可以在每个Pass之后打印出IR,如果它被那个Pass修改过。使用上一段中的\texttt{test.c}(及其IR文件\texttt{test.ll})示例代码,我们可以使用以下命令打印每个Pass所做的更改(如果有更改的话):

\begin{tcblisting}{commandshell={}}
$ opt -O2 --disable-output --print-changed ./test.ll
*** IR Dump At Start: ***
...
define dso_local i32 @bar(i32 %x) #0 {
  entry:
  %x.addr = alloca i32, align 4
\end{tcblisting}

\begin{tcblisting}{commandshell={}}
  %y = alloca i32, align 4
  …
  %1 = load i32, i32* %y, align 4
  %mul = mul nsw i32 %1, 4
  ret i32 %mul
}
...
*** IR Dump After VerifierPass (module) omitted because no
change ***
…
...
*** IR Dump After SROA *** (function: bar)
; Function Attrs: noinline nounwind uwtable
define dso_local i32 @bar(i32 %x) #0 {
  entry:
  %mul = mul nsw i32 %x, 4
  ret i32 %mul
}
...
$
\end{tcblisting}

这里,我们只显示了少量的输出。在代码中,我们可以看到该工具将首先打印出原始IR(\texttt{IR Dump At Start}),然后在每个Pass处理后显示IR,例如:前面的代码显示,在SROA Pass之后,\texttt{bar}函数变短了。如果Pass没有修改IR,将忽略IR转储,以减少噪声的数量。

有时,我们只对发生在特定函数集上的变化感兴趣,例如:本例中的\texttt{foo}函数。我们可以添加\texttt{-\,-filter-print-funcs=<函数名>}标志(而不是打印整个模块的更改日志),从而仅打印函数子集的IR更改,例如:只打印\texttt{foo}函数的IR更改,可以使用以下命令:

\begin{tcblisting}{commandshell={}}
$ opt -O2 --disable-output \
          --print-changed --filter-print-funcs=foo ./test.ll
\end{tcblisting}

就像\texttt{-\,-filter-print-funcs},有时我们只希望看到由特定的一组传递所做的更改,比如:SROA和\texttt{InstCombine} Pass。这种情况下,可以添加\texttt{-\,-filter-passes=<Pass 名称>},例如:只查看与SROA和\texttt{InstCombine}相关的内容,我们可以使用以下命令:

\begin{tcblisting}{commandshell={}}
$ opt -O2 --disable-output \
          --print-changed \
          --filter-passes=SROA,InstCombinePass ./test.ll
\end{tcblisting}

现在,已经了解了如何打印流水中所有Pass之间的IR差异,使用额外的过滤器可以进一步关注特定的功能或Pass。换句话说,这个工具可以轻松地观察Pass流水中IR的变化,并快速地发现可能感兴趣的任何痕迹。下一节中,我们将了解如何通过将Pass流水一分为二来调试代码优化中出现的问题。

\subsubsubsection{9.4.3\hspace{0.2cm}将Pass流水一分为二}

上一节中,介绍了\texttt{-\,-print-changed}标志,它在Pass流水中打印出IR更改日志。我们还提到,对于感兴趣的IR改变这个标志是有用的,例如:导致错误编译错误的无效代码转换。或者,也可以将Pass流水一分为二,以实现类似的目的。更具体地说,可以使用\texttt{opt}中的\texttt{-\,-opt-bisect-limit=<n>}标志,通过禁用除前\texttt{N}个以外的所有Pass,来分割Pass管道。下面的命令展示了一个这样的例子:

\begin{tcblisting}{commandshell={}}
$ opt -O2 --opt-bisect-limit=5 -S -o – test.ll
BISECT: running pass (1) Annotation2MetadataPass on module (./test.ll)
BISECT: running pass (2) ForceFunctionAttrsPass on module (./test.ll)
BISECT: running pass (3) InferFunctionAttrsPass on module (./test.ll)
BISECT: running pass (4) SimplifyCFGPass on function (bar)
BISECT: running pass (5) SROA on function (bar)
BISECT: NOT running pass (6) EarlyCSEPass on function (bar)
BISECT: NOT running pass (7) LowerExpectIntrinsicPass on function (bar)
BISECT: NOT running pass (8) SimplifyCFGPass on function (foo)
BISECT: NOT running pass (9) SROA on function (foo)
BISECT: NOT running pass (10) EarlyCSEPass on function (foo)
...
define dso_local i32 @bar(i32 %x) #0 {
  entry:
  %mul = mul nsw i32 %x, 4
  ret i32 %mul
}
define dso_local i32 @foo(i32 %y) #0 {
  entry:
  %y.addr = alloca i32, align 4
  store i32 %y, i32* %y.addr, align 4
\end{tcblisting}

\begin{tcblisting}{commandshell={}}
  %0 = load i32, i32* %y.addr, align 4
  %1 = load i32, i32* %y.addr, align 4
  %mul = mul nsw i32 %1, 2
  %add = add nsw i32 %0, %mul
  ret i32 %add
} $
\end{tcblisting}

(请注意,这与前面几节中的例子不同,上面的命令同时打印了\texttt{-\,-opt-bisect-limit}和最后的IR。)

由于我们实现了\texttt{-\,-opt-bisect-limit=5}标志,Pass流水只运行前5个Pass。正如从诊断消息中看到的,SROA应用于\texttt{bar},而不是\texttt{foo}函数,这使得\texttt{foo}的最终IR不是最优的。

通过改变\texttt{-\,-opt-bisect-limit}后面的数字,我们可以调整切割点,直到出现某些代码更改或触发某个bug(例如,崩溃)。这是非常有用的(前期)过滤步骤,可以将原始问题缩小到更小的范围。此外,由于它使用数值作为参数,这个特性非常适合于自动化环境,例如:自动崩溃报告工具或性能回归跟踪工具。

本节中,介绍了几种用于调试和诊断Pass流水的工具。当涉及到修复问题(如编译器崩溃、(目标程序上的)性能倒退和编译错误)时,这些工具可以极大地提高工作效率。































