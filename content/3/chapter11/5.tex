
错误处理一直是软件开发中广泛讨论的话题。可以像返回错误代码一样简单——例如:在许多Linux API中(例如,\texttt{open}函数)——或者使用高级机制,例如:抛出异常,这已经被许多现代编程语言(如Java和C++)广泛采用。

虽然C++内置了对异常处理的支持,但LLVM在其代码库中根本没有使用。这一决定背后的基本原因是,尽管方便表达语法,但C++中的异常处理在性能方面需要付出了很高的代价。简单地说,异常处理使原始代码更加复杂,并妨碍编译器对其进行优化。此外,在运行时,程序通常需要花费更多的时间从异常中恢复。因此,LLVM在其代码库中默认禁用异常处理,转而使用其他错误处理方法——例如,在返回值中携带一个错误或使用本节中的工具。

本节的前半部分,我们将讨论\texttt{Error}类,它代表错误。这与传统的错误表示不同——例如,当使用整数作为错误代码时,不能忽略生成的\texttt{Error}实例而不处理。我们稍后将对此进行解释。

除了\texttt{Error}类,在LLVM的代码库中,许多错误处理代码都有一个共同的模式:API可以返回结果或错误,但不能同时返回结果或错误。例如,当我们调用读取文件的API时,期望得到该文件的内容(结果)或出错时的错误(例如,没有这样的文件)。本节的第二部分中,我们将了解两个实现此模式的工具类。

首先介绍\texttt{Error}类。

\subsubsubsection{11.5.1\hspace{0.2cm}Error类}

\texttt{Error}类所表示的概念非常简单:一个带有附加描述的错误,比如错误消息或错误代码,通过一个值(作为函数的参数)传递或从函数返回。开发人员也可以创建自己的\texttt{Error}实例,例如:如果想创建一个\texttt{FileNotFoundError}实例来告诉用户某个文件不存在,代码如下:

\begin{lstlisting}[style=styleCXX]
#include "llvm/Support/Error.h"
#include <system_error>
// In the header file…
struct FileNotFoundError : public ErrorInfo<FileNoteFoundError>
{
	StringRef FileName;
	explicit FileNotFoundError(StringRef Name) : FileName(Name)
	{}
	static char ID;
	std::error_code convertToErrorCode() const override {
		return std::errc::no_such_file_or_directory;
	}
	void log(raw_ostream &OS) const override {
		OS << FileName << ": No such file";
	}
};
// In the CPP file…
char FileNotFoundError::ID = 0;
\end{lstlisting}

实现自定义\texttt{Error}实例有几个要求:

\begin{itemize}
\item 从\texttt{ErrorInfo<T>}类派生,其中\texttt{T}是自定义类型。
\item 声明唯一的\texttt{ID}变量。本例中,我们使用一个静态类成员变量。
\item 实现\texttt{convertToErrorCode},这个方法为\texttt{Error}实例指定\texttt{std::error\_code}实例。\texttt{std::err or\_code}是C++标准库中使用的错误类型(从C++11开始)。请参考C++参考文档获取(预定义的)\texttt{std::error\_code}的实例。
\item 实现\texttt{log}的方式打印出错误消息。
\end{itemize}

要创建\texttt{Error}实例,可以使用\texttt{make\_error}函数:

\begin{lstlisting}[style=styleCXX]
Error NoSuchFileErr = make_error<FileNotFoundError>("foo.txt");
\end{lstlisting}

\texttt{make\_error}函数接受一个\texttt{Error}类(在本例中是\texttt{FileNotFoundError}类)作为模板参数和函数参数(在本例中是\texttt{foo.txt})。这些将传递给其构造函数。

如果尝试运行上面的代码(在调试构建中),而不对\texttt{NoSuchFileErr}变量做任何操作,程序将会崩溃,并显示如下错误消息:

\begin{tcblisting}{commandshell={}}
Program aborted due to an unhandled Error:
foo.txt: No such file
\end{tcblisting}

结果是,每个\texttt{Error}实例都需要在其生命周期结束之前(即在调用其析构函数方法时)进行检查和处理。

先解释一下检查\texttt{Error}实例意味着什么。除了表示真实的错误外,\texttt{Error}类还可以表示成功状态,即没有错误。为了更具体地了解这一点,许多LLVM API都有以下错误处理的结构:

\begin{lstlisting}[style=styleCXX]
Error readFile(StringRef FileName) {
	if (openFile(FileName)) {
		// Success
		// Read the file content…
		return ErrorSuccess();
	} else
	return make_error<FileNotFoundError>(FileName);
}
\end{lstlisting}

注意,百分之百确定它处于成功状态,也要需要检查\texttt{Error}实例,否则程序仍然会中止运行。

前面的代码段提供了一个很好过渡到处理\texttt{Error}实例的例程。如果一个\texttt{Error}实例代表了一个真实的错误,就需要使用一个特殊的API来处理它:\texttt{handleErrors}。下面是如何使用它:

\begin{lstlisting}[style=styleCXX]
Error E = readFile(…);
if (E) {
	// TODO: Handle the error
} else {
    // Success!
}
\end{lstlisting}

\texttt{handleErrors}函数获得\texttt{Error}实例的所有权(通过\texttt{std::move(E)}),并使用提供的lambda函数来处理错误。我们注意到\texttt{handleErrors}会返回一个\texttt{Error}实例,来表示未处理的错误。这是什么意思呢?

前面的\texttt{readFile}函数示例中,返回的\texttt{Error}实例,既可以表示\texttt{Success}状态,也可以表示\texttt{FileNotFoundError}状态。我们可以稍微修改这个函数,当打开的文件为空时,返回一个\texttt{FileEmpty Error}实例:

\begin{lstlisting}[style=styleCXX]
Error E = readFile(…);
if (E) {
	Error UnhandledErr = handleErrors(
	std::move(E),
	[&](const FileNotFoundError &NotFound) {
		NotFound.log(errs() << "Error occurred: ");
		errs() << "\n";
	});
	…
}
\end{lstlisting}

现在,\texttt{readFile}返回的\texttt{Error}实例可以是一个\texttt{Success}状态,或一个\texttt{FileNotFoundError}实例,再或一个\texttt{FileEmptyError}实例。然而,我们之前写的\texttt{handleErrors}代码,只能处理\texttt{FileNotFoundError}的情况.

因此,我们需要使用以下代码来处理\texttt{FileEmptyError}的情况:

\begin{lstlisting}[style=styleCXX]
Error readFile(StringRef FileName) {
	if (openFile(FileName)) {
		// Success
		…
		if (Buffer.empty())
			return make_error<FileEmptyError>();
		else
			return ErrorSuccess();
	} else
		return make_error<FileNotFoundError>(FileName);
}
\end{lstlisting}

注意,在使用\texttt{handleErrors}时,需要获取\texttt{Error}实例的所有权。

或者,可以通过为每种错误类型使用多个lambda函数参数将两个\texttt{handleErrors}函数调用合并为一个:

\begin{lstlisting}[style=styleCXX]
Error E = readFile(…);
if (E) {
	Error UnhandledErr = handleErrors(
	std::move(E),
	[&](const FileNotFoundError &NotFound) {…},
	[&](const FileEmptyError &IsEmpty) {…});
	…
}
\end{lstlisting}

换句话说,\texttt{handleErrors}函数的作用类似于\texttt{Error}实例的switch语句,运行起来像下面的伪代码一样:

\begin{lstlisting}[style=styleCXX]
Error E = readFile(…);
if (E) {
	switch (E) {
		case FileNotFoundError: …
		case FileEmptyError: …
		default:
			// generate the UnhandledError
	}
}
\end{lstlisting}

现在,由于\texttt{handleErrors}总是返回一个表示未处理错误的\texttt{Error},就不能忽略返回的实例,否则程序将中止运行,这时我们应该如何结束这个“错误处理链”?有两种方法,让我们来了解一下:

\begin{itemize}
\item 如果100\%确信已经处理了所有可能的错误类型——这意味着未处理的\texttt{Error}变量处于\texttt{Success}状态——可以使用\texttt{cantFail}函数进行断言:

\begin{lstlisting}[style=styleCXX]
if (E) {
	Error UnhandledErr = handleErrors(
		std::move(E),
		[&](const FileNotFoundError &NotFound) {…},
		[&](const FileEmptyError &IsEmpty) {…});
	cantFail(UnhandledErr);
}
\end{lstlisting}

如果\texttt{UnhandledErr}仍然包含一个错误,\texttt{cantFail}函数将中止程序执行,并打印一条错误消息。

\item 一个更优雅的解决方案是使用\texttt{handleAllErrors}函数:

\begin{lstlisting}[style=styleCXX]
if (E) {
	handleAllErrors(
	std::move(E),
	[&](const FileNotFoundError &NotFound) {…},
	[&](const FileEmptyError &IsEmpty) {…});
	…
}
\end{lstlisting}

这个函数不会返回任何东西,假设给定的lambda函数足以处理所有可能的错误类型。当然,如果缺少,\texttt{handleAllErrors}仍然会中止程序。

\end{itemize}

现在,我们已经了解了如何使用\texttt{Error}类,以及如何正确处理错误。虽然\texttt{Error}的设计乍一看似乎有些烦(也就是说,我们需要处理所有可能的错误类型,否则执行将中途中止),但这些限制可以减少程序员犯的错误数量,并创建更健壮的程序。

接下来,我们将介绍另外两个工具类,它们可以进一步提升LLVM中的错误处理表达式。














