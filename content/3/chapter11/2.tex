在软件开发中,有许多方法可以诊断错误——例如,使用调试器、在程序中插入杀毒器(例如,捕获无效的内存访问),或者简单地使用最简单,但最有效的方法之一:添加打印语句。虽然最后一个选项听起来不是很聪明,但在其他选项不能充分发挥其潜力的许多情况下(例如,调试信息质量较差的发布模式二进制文件或多线程程序),其实非常有用。

LLVM提供了一个小工具,不仅可以打印调试消息,还过滤要显示的消息。假设我们有一个LLVM Pass——\texttt{SimpleMulOpt}——用左移操作替换2的幂次乘法(这是我们在前一章处理LLVM IR的最后一节中所完成的)。下面是它的运行方法的部分实现:

\begin{lstlisting}[style=styleCXX]
PreservedAnalyses
SimpleMulOpt::run(Function &F, FunctionAnalysisManager &FAM) {
	for (auto &I : instructions(F)) {
		if (auto *BinOp = dyn_cast<BinaryOperator>(&I) &&
		BinOp->getOpcode() == Instruction::Mul) {
			auto *LHS = BinOp->getOperand(0),
			*RHS = BinOp->getOperand(1);
			// `BinOp` is a multiplication, `LHS` and `RHS` are its
			// operands, now trying to optimize this instruction…
			…
		}
	}
	…
}
\end{lstlisting}

前面的代码遍历给定函数中的所有指令,然后查找表示算术乘法的指令。如果有,Pass将与LHS和RHS操作数一起工作(它们出现在代码的其余部分中——这里没有显示)。

假设我们想在开发过程中打印出操作数变量。大多数原生的方法是使用我们的“老朋友”\texttt{errs()},它可以将任意消息发送到\texttt{stderr}。如下所示:

\begin{lstlisting}[style=styleCXX]
// (extracted from the previous snippet)
…
auto *LHS = BinOp->getOperand(0),
     *RHS = BinOp->getOperand(1);
errs() << "Found a multiplication with operands ";
LHS->printAsOperand(errs());
errs() << " and ";
RHS->printAsOperand(errs());
…
\end{lstlisting}

代码中使用的\texttt{printAsOperand}将\texttt{Value}的文本表示输出到给定流中(在本例中为\texttt{errs()})。

一切看起来都很正常,除了这些消息无论如何都会被打印出来。不过是在生产环境中,这不是我们想要的。要么需要在交付产品之前删除这些代码,要么在这些代码周围添加一些宏保护(例如,\texttt{\#ifndef NDEBUG}),要么可以使用LLVM提供的调试工具。下面是一个例子:

\begin{lstlisting}[style=styleCXX]
#include "llvm/Support/Debug.h"
#define DEBUG_TYPE "simple-mul-opt"
…
auto *LHS = BinOp->getOperand(0),
     *RHS = BinOp->getOperand(1);
LLVM_DEBUG(dbgs() << "Found a multiplication with operands ");
LLVM_DEBUG(LHS->printAsOperand(dbgs()));
LLVM_DEBUG(dbgs() << " and ");
LLVM_DEBUG(RHS->printAsOperand(dbgs()));
…
\end{lstlisting}

上面的代码基本上做了以下三件事:

\begin{itemize}
\item 用\texttt{dbgs()}替换\texttt{errors()}。这两个流基本上做的是相同的事情,但后者将向输出消息添加一个漂亮的格式(调试日志输出)。

\item 使用\texttt{LLVM\_DEBUG(…)}宏函数包装所有与调试打印相关的行。使用这个宏可以确保只在开发模式下编译封闭行。它还对调试消息类别进行编码,稍后我们将介绍这个类别。

\item 使用\texttt{LLVM\_DEBUG(…)}宏函数之前,请确保您将\texttt{DEBUG\_TYPE}定义为所需的调试类别字符串(本例中为\texttt{simple-mult-opt})。

\end{itemize}

除了上述代码修改之外,我们还需要使用一个额外的命令行标志\texttt{-debug},可以选择打印那些调试消息。下面是一个例子:

\begin{tcblisting}{commandshell={}}
$ opt -O3 -debug -load-pass-plugin=… …
\end{tcblisting}

但是,会发现输出非常嘈杂。有大量来自其他LLVM PAss的调试消息。在本例中,我们只对来自我们Pass的消息感兴趣。

要过滤掉不相关的消息,可以使用\texttt{-debug-only}命令行标志。下面是一个例子:

\begin{tcblisting}{commandshell={}}
$ opt -O3 -debug-only=simple-mul-opt -load-pass-plugin=… …
\end{tcblisting}

\texttt{-debug-only}后面的值是我们在前面的代码片段中定义的\texttt{DEBUG\_TYPE}值。换句话说,我们可以使用由每个Pass定义\texttt{的DEBUG\_TYPE}来过滤所需的调试消息。我们还可以选择要打印的多个调试类别,使用以下命令:

\begin{tcblisting}{commandshell={}}
$ opt -O3 -debug-only=sroa,simple-mul-opt -load-pass-plugin=… …
\end{tcblisting}

这个命令不仅打印来自\texttt{SimpleMulOpt} Pass的调试消息,还打印来自\texttt{SROA} Pass的调试消息——包含在\texttt{O3}优化管道中的LLVM Pass。

除了为LLVM Pass定义一个单独的调试类别(\texttt{DEBUG\_TYPE})之外,实际上可以自由地在Pass中使用任意多的类别,例如:当希望对Pass的不同部分使用单独的调试类别时,这是很有用的。我们可以在\texttt{SimpleMulOpt} Pass中为每个操作数使用单独的类别。可以这样做:

\begin{lstlisting}[style=styleCXX]
…
#define DEBUG_TYPE "simple-mul-opt"
auto *LHS = BinOp->getOperand(0),
     *RHS = BinOp->getOperand(1);
LLVM_DEBUG(dbgs() << "Found a multiplication instruction");
DEBUG_WITH_TYPE("simple-mul-opt-lhs",
               LHS->printAsOperand(dbgs() << "LHS operand: "));
DEBUG_WITH_TYPE("simple-mul-opt-rhs",
               RHS->printAsOperand(dbgs() << "RHS operand: "));
…
\end{lstlisting}

\texttt{DEBUG\_WITH\_TYPE}是\texttt{LLVM\_DEBUG}的特殊版本。在第二个参数处执行代码,第一个参数作为调试类别,它可以不同于当前定义的\texttt{DEBUG\_TYPE}值。前面的代码中,除了使用\texttt{simplemul-opt}类别打印\texttt{Found a multiplication instruction}外,我们还使用\texttt{simple-multi-opt-lhs}打印与\textbf{左侧(LHS)}操作数相关的消息,并使用\texttt{simple-mul-opt-rhs}打印其他操作数的消息。有了这个特性,就可以更细粒度地通过\texttt{opt}命令选择调试消息类别了。

现在,已经了解了如何使用LLVM提供的实用程序仅在开发环境中打印调试消息,以及如何在需要时对它们进行过滤。下一节中,我们将了解如何在运行LLVM Pass时收集关键统计信息。








