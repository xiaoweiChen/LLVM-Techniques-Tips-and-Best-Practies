本章中,我们通过处理静态源代码和捕获程序的运行时行为来扩充编译器的工作空间。本章的第一部分中,我们了解了如何使用LLVM提供的工具来创建一个杀毒器——一种将检测代码插入到目标程序中以检查特定运行时属性的技术。通过使用杀毒器,软件工程师可以轻松和高精度地提高他们的开发质量。本章的第二部分,我们将这些运行时数据的使用扩展到编译器优化的领域——PGO是一种使用动态信息(如基本块或函数的执行频率)来做出更积极的优化代码的决策的技术。最后,我们了解了如何使用LLVM Pass访问这些数据,这使我们能够使用PGO强化现有的优化策略。

恭喜你,读完了最后一章!非常感谢您阅读这本书。在计算机科学中,编译器开发从来都不是一门简单的学科——如果不是晦涩难懂的话。在过去的十年里,LLVM提供了健壮而灵活的模块,从根本上改变了人们对编译器的看法,从而大大降低了这一课题的难度。编译器不再只是像\texttt{gcc}或\texttt{clang}那样的单个可执行文件——它是一组构建块的集合,为开发人员提供了无数种方法创建工具,来处理编程语言领域的难题。

然而,选择如此之多,当我还是LLVM新手的时候,我经常会迷失和困惑。这个项目中的每个API都有文档,但我不知道如何将它们组合在一起。我希望有一本书能指出LLVM中每个重要组件的大致方向,告诉我它是什么,以及我如何利用它。这就是我在LLVM职业生涯开始时希望得到的那本书——你刚刚读完的那本书。我希望你读完这本书后不要停止你的LLVM之旅,了进一步提高你的技能,并加强你从这本书中学到的东西,我建议你查看官方文档页面(\url{https://llvm.org/docs}),以获得补充这本书的内容。更重要的是,我鼓励你通过邮件列表(\url{https://lists.llvm.org/cgi-bin/mailman/listinfo/llvmdev})或话语论坛(\url{https://llvm.discourse.group/})参与LLVM社区,特别是第一个论坛——尽管邮件列表可能听起来很老派,那里有很多有才华的人愿意回答你的问题,并提供有用的学习资源。最后,每年的LLVM开发会议(\url{https://llvm.org/devmtg/}),在美国和欧洲(有着非常最好的活动),在这里你可以学习新的LLVM技能,并与真正维护LLVM的人面对面聊天。

我希望这本书在掌握LLVM的道路上对你有所启发,并帮助你找到制作编译器的乐趣。







