In this chapter, we augmented the workspace of the compiler by processing the static source code and capturing the program's runtime behaviors. In the first part of this chapter, we learned how to use the infrastructure provided by LLVM to create a sanitizer – a technique that inserts instrumentation code into the target program for the sake of checking certain runtime properties. By using a sanitizer, software engineers can improve their development quality with ease and with high precision. In the second part of this chapter, we extended the usages of such runtime data to the domain of compiler optimization; PGO is a technique that uses dynamic information, such as the execution frequency of basic blocks or functions, to make more aggressive decisions for optimizing the code. Finally, we learned how to access such data with an LLVM Pass, which enables us to add PGO enhancement to existing optimizations.

Congratulations, you've just finished the last chapter! Thank you so much for reading this book. Compiler development has never been an easy subject – if not an obscure one – in computer science. In the past decade, LLVM has significantly lowered the difficulties of this subject by providing robust yet flexible modules that fundamentally change how people think about compilers. A compiler is not just a single executable such as gcc or clang anymore – it is a collection of building blocks that provide developers with countless ways to create tools to deal with hard problems in the programming language field.

However, with so many choices, I often became lost and confused when I was still a newbie to LLVM. There was documentation for every single API in this project, but I had no idea how to put them together. I wished there was a book that pointed in the general direction of each important component in LLVM, telling me what it is and how I can take advantage of it. And here it is, the book I wished I could have had at the beginning of my LLVM career – the book you just finished – come to life. I hope you won't stop your expedition of LLVM after finishing this book. To improve your skills even further and reinforce what you've learned from this book, I recommend you to check out the official document pages (\url{https://llvm.org/docs}) for content that complements this book. More importantly, I encourage you to participate in the LLVM community via either their mailing list (\url{https://lists.llvm.org/cgi-bin/mailman/listinfo/llvmdev}) or Discourse forum (\url{https://llvm.discourse.group/}), especially the first one – although a mailing list might sound old-school, there are many talented people there willing to answer your questions and provide useful learning resources. Last but not least, annual LLVM dev meetings (\url{https://llvm.org/devmtg/}), in both the United States and Europe, are some of the best events where you can learn new LLVM skills and chat face-to-face with people who literally built LLVM.

I hope this book enlightened you on your path to mastering LLVM and helped you find joy in crafting compilers.







