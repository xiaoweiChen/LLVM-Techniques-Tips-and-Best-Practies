前一章中,我们了解了如何在使用LLVM进行开发时利用各种工具来提高工作效率,这些技能可以让我们在诊断出LLVM的问题时,处理起来更加顺畅。其中一些工具可以减少编译器工程师犯错的机会。本章中,我们将了解在LLVM IR中工具是如何工作的。

我们在这里所指的\textbf{工具}是一种技术,将一些检测插入到我们正在编译的代码中,以收集运行时信息,例如:可以收集关于某个函数调用多少次的信息——只有在目标程序执行之后才可用。这种技术的优点是提供了关于目标程序行为的准确信息,这些信息可以以几种不同的方式使用,例如:可以使用收集到的值,再次编译和优化相同的代码——但是这一次,由于有准确的数据,可以执行以前无法执行的优化。这种技术也称为\textbf{数据导向优化(Profile-Guided Optimization,PGO)}。在另一个例子中,将使用插入的检测来捕获运行时不希望发生的事件——缓冲区溢出、条件竞争和双重释放内存等等。用于此目的的检测器,也称为\textbf{杀毒器}。

要在LLVM中实现相应的工具,不仅需要LLVM Pass的帮助,还需要LLVM-\textbf{Clang}、\textbf{LLVM IR} \textbf{Transformation}和\textbf{Compiler-RT}中的多个子项目之间的协同。在本章中,我们将介绍Compiler-RT,更重要的是,了解如何将这些子系统组合在一起,从而达到测试的目的。

下面是我们将要讨论的内容:

\begin{itemize}
\item 开发杀毒器
\item 使用PGO
\end{itemize}

本章的第一部分,将看到如何在Clang和LLVM中实现杀毒器,然后自己创建一个简单的杀毒器。本章的后半部分将展示如何在LLVM中使用PGO框架,以及如何扩展它。






