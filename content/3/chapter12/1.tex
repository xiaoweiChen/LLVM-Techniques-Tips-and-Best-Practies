In this chapter, we are going to work with multiple subprojects. One of them – Compiler-RT – needs to be included in your build by us modifying the CMake configuration. Please open the CMakeCache.txt file in your build folder and add the compiler-rt string to the value of the LLVM\_ENABLE\_PROJECTS variable. Here is an example:

\begin{lstlisting}[style=styleCMake]
//Semicolon-separated list of projects to build…
LLVM_ENABLE_PROJECTS:STRING="clang;compiler-rt"
\end{lstlisting}

After editing the file, launch a build with any build target. CMake will try to reconfigure itself.

Once everything has been set up, we can build the components we need for this chapter.Here is an example command:

\begin{tcblisting}{commandshell={}}
$ ninja clang compiler-rt opt llvm-profdata
\end{tcblisting}

This will build the clang tool we're all familiar with and a collection of Compiler-RT libraries, which we are going to introduce shortly.

You can find the sample code for this chapter in the same GitHub repository: \url{https://github.com/PacktPublishing/LLVM-Techniques-Tips-and-Best-Practices-Clang-and-Middle-End-Libraries/tree/main/Chapter12}.





















