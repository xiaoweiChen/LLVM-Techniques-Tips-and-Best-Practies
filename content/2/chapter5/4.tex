In this chapter, we looked at how Clang is organized and the functionalities of some of its important subsystems and components. Then, we learned about the differences between Clang's major extension and tooling options – the Clang plugin, libTooling, and Clang Tools – including what each of them looks like and what their pros and cons are. The Clang plugin provides an easy way to insert custom logic into Clang's compilation pipeline via dynamically loaded plugins but suffers from API stability issues; libTooling has a different focus than the Clang plugin in that it aims to provide a toolbox for developers to create a standalone tool; and Clang Tools provides various applications.

In the next chapter, we will talk about preprocessor development. We will learn how the preprocessor and the lexer work in Clang, and show you how to write plugins for the sake of customizing preprocessing logic.