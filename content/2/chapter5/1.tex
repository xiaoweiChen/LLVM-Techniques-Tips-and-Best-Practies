第1章中,了解了如何构建LLVM。不过,没有构建Clang。要将Clang包含在构建列表中,需要修改CMake变量\texttt{LLVM\_ENABLE\_PROJECTS}的值,如下所示:

\begin{tcblisting}{commandshell={}}
$ cmake -G Ninja -DLLVM_ENABLE_PROJECTS="clang;clang-toolsextra" …
\end{tcblisting}

该变量的值是以分号分隔的列表,其中每一项都是LLVM的子项目。本例包括了Clang和\texttt{clang-tools-extra},其中包含了一组基于Clang的有用工具,例如:\texttt{clang-format}可以对编码风格进行统一,已经在无数的开源项目使用,特别是大型项目。

\begin{tcolorbox}[colback=blue!5!white,colframe=blue!75!black, fonttitle=\bfseries,title=将Clang添加到现有的构建中]
\hspace*{0.7cm}如果已经有了未启用Clang的LLVM构建,可以在CMakeCache.txt中编辑CMake参数\texttt{LLVM\_ENABLE\_PROJECTS},而不需要再次使用CMake命令。编辑了文件,并再次运行Ninja(或你选择的构建系统),CMake就会重新配置。
\end{tcolorbox}

可以使用以下命令来构建\texttt{clang}, Clang的驱动程序和主程序:

\begin{tcblisting}{commandshell={}}
$ ninja clang
\end{tcblisting}

可以使用以下命令运行所有Clang测试:

\begin{tcblisting}{commandshell={}}
$ ninja check-clang
\end{tcblisting}

现在,\texttt{clang}可执行文件应该就在\texttt{/<your build directory>/bin}文件夹中。



