Clang是LLVM的C族编程语言的官方前端,包括C、C++和Objective-C。处理输入源代码(解析、类型检查和语义推理等),并生成等效的LLVM IR代码,然后过其他LLVM子系统,执行优化和/或本地代码生成。许多类似C语言的方言或语言扩展也开始使用Clang托管它们的实现,例如:Clang为OpenCL、OpenMP和CUDA C/C++提供官方支持。除了常规的前端工作外,Clang一直在发展,将其功能划分为库和模块,这样开发人员可以使用它们来创建各种与源代码处理相关的工具,例如:代码重构、代码格式化和语法高亮显示。学习Clang开发不仅可以让您更多地参与到LLVM项目中,还可以为创建功能强大的应用程序和工具提供了更多的可能性。

LLVM将其大部分任务安排在管道(即PassManager)中顺序执行。与LLVM不同,Clang在组织子组件的方式上更加多样化。在本章中,将向您展示Clang的系统是如何组织的,角色分别是什么,以及其在源码的哪一部分。

\begin{tcolorbox}[colback=blue!5!white,colframe=blue!75!black, fonttitle=\bfseries,title=术语]
\hspace*{0.7cm}这一章到本书的其余部分,将使用Clang(以大写C和Minion Pro字体开头)作为整体来引用项目和使用到的技术。当使用clang时(小写,Courier字体),则指的是可执行程序。
\end{tcolorbox}

本章中,我们将讨论以下内容:

\begin{itemize}
\item 了解Clange的子系统及其作用
\item 探索Clang工具的功能和扩展选项
\end{itemize}

本章结束时,您将对这个系统的路线图有了一定的了解,这样您就可以启动您感兴趣的项目,并为以后有关Clang开发的提供一些参考。


