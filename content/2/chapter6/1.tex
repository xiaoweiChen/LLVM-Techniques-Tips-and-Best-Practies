This chapter expects you to have a build of the Clang executable. You can obtain this by
running the following command:

\begin{tcblisting}{commandshell={}}
$ ninja clang
\end{tcblisting}

Here is a useful command to print textual content right after preprocessing:

The -E command-line option for clang is pretty useful for printing textual content right after preprocessing. As an example, foo.c has the following content:

\begin{lstlisting}[style=styleCXX]
#define HELLO 4
int foo(int x) {
	return x + HELLO;
}
\end{lstlisting}

Use the following command:

\begin{tcblisting}{commandshell={}}
$ clang -E foo.c
\end{tcblisting}

The preceding command will give you this output:

\begin{lstlisting}[style=styleCXX]
…
int foo(int x) {
	return x + 4;
}
\end{lstlisting}

As you can see, HELLO was replaced by 4 in the code. You might be able to use this trick to debug when developing custom extensions in later sections.

Code used in this chapter can be found at this link: \url{https://github.com/PacktPublishing/LLVM-Techniques-Tips-and-Best-Practices-Clangand-Middle-End-Libraries/tree/main/Chapter06}.









