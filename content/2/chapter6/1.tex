本章希望您能够构建Clang可执行文件,可以通过以下命令获取:

\begin{tcblisting}{commandshell={}}
$ ninja clang
\end{tcblisting}

下面的命令,可以在预处理后立即打印文本内容:

\texttt{clang}的\texttt{-E}选项对于在预处理之后立即打印文本内容非常有用。例如,\texttt{foo.}c包含如下内容:

\begin{lstlisting}[style=styleCXX]
#define HELLO 4
int foo(int x) {
	return x + HELLO;
}
\end{lstlisting}

然后使用如下命令:

\begin{tcblisting}{commandshell={}}
$ clang -E foo.c
\end{tcblisting}

上面的命令会给出这样的输出:

\begin{lstlisting}[style=styleCXX]
…
int foo(int x) {
	return x + 4;
}
\end{lstlisting}

如您所见,代码中的\texttt{HELLO}被4替换了。后面,可以在开发自定义扩展时使用此技巧进行调试。

本章的代码链接: \url{https://github.com/PacktPublishing/LLVM-Techniques-Tips-and-Best-Practices-Clangand-Middle-End-Libraries/tree/main/Chapter06}.









