前一章中,我们讨论了Clang的结构——C语言\textbf{底层虚拟机(LLVM)}的官方前端——以及它的一些最重要的组件。还介绍了各种Clang的工具和扩展选项。本章中,将深入到Clang前端管道的第一阶段:预处理器。

\textbf{预处理}是C族编程语言的一个早期编译阶段,会替换了任何以哈希(\#)字符开始的指令——\texttt{\#include}和\texttt{\#define}——只命名一个少数的文本内容(或非文本标记)。例如,在解析头文件之前,预处理器基本上会将由\texttt{\#include}指令指定的头文件的内容复制粘贴到当前编译单元中。这种技术的优点是可以提取通用代码进行重用。

本章中,将简要了解Clang的\textbf{预处理器/词法分析器}框架是如何工作的,以及一些重要的\textbf{应用程序编程接口(API)},这些接口可以帮助开发者进行开发。此外,Clang还提供了一些方法,让开发者可以通过插件将自定义逻辑注入到预处理流程中,例如:允许以一种更简单的方式创建自定义的\texttt{\#pragma}语法——OpenMP使用的语法(例如,\texttt{\#pragma omp}循环)。了解这些技术可以在解决不同抽象级别的问题时有更多的选择。以下是本章各小节的列表:

\begin{itemize}
\item 使用\texttt{SourceLocation}和\texttt{SourceManager}
\item 了解预处理器和词法分析器的基本知识
\item 定制开发预处理器的插件和回调
\end{itemize}

















