
Clang的预处理框架也提供通过插件插入自定义逻辑的方法,与LLVM和Clang的其他部分一样灵活。更具体地说,允许开发人员编写插件来处理定制的\textbf{pragma}指令(允许用户编写诸如\texttt{\#pragma my\_awesome\_feature}之类的东西)。此外,\texttt{Preprocessor}类还提供了一种更通用的方式来定义回调函数,以响应任意的\textbf{预处理事件}——例如,当宏展开或\texttt{\#include}指令解析时。本节中,我们将使用一个简单的项目,利用这两种技术来展示它们的用法。

\subsubsubsection{6.4.1\hspace{0.2cm}项目的目标和准备工作}

C/C++中的宏以不卫生的设计而臭名昭著,如果使用不当,很容易导致编码错误。请看下面的代码片:

\begin{lstlisting}[style=styleCXX]
#define PRINT(val) \
printf("%d\n", val * 2)
void main() {
	PRINT(1 + 3);
}
\end{lstlisting}

前面代码中的\texttt{PRINT}看起来就像一个普通的函数,因此很容易相信这个程序将输出8。然而这里,\texttt{PRINT}是一个宏函数而不是普通函数,所以当它展开时,main函数等价于:

\begin{lstlisting}[style=styleCXX]
void main() {
	printf("%d\n", 1 + 3 * 2);
}
\end{lstlisting}

因此,程序实际输出7。这种歧义当然可以通过在宏体中每次出现\texttt{val}宏参数时都用括号括起来来解决,如下面的代码所示:

\begin{lstlisting}[style=styleCXX]
#define PRINT(val) \
printf("%d\n", (val) * 2)
\end{lstlisting}

现在,宏展开后,主函数将是这样的:

\begin{lstlisting}[style=styleCXX]
void main() {
	printf("%d\n", (1 + 3) * 2);
}
\end{lstlisting}

我们在这里要做的项目是开发一个自定义的\texttt{\#pragma}语法来警告开发人员,如果某个由程序员指定的宏参数没有正确地括在括号中,以防止前面的卫生问题发生。下面是这个新语法的例子:

\begin{lstlisting}[style=styleCXX]
#pragma macro_arg_guard val
#define PRINT(val) \
printf("%d\n", val * 94 + (val) * 87);
void main() {
	PRINT(1 + 3);
}
\end{lstlisting}

与前面的例子类似,如果前面的\texttt{val}参数没有出现在括号中,这会引入bug。

新的\texttt{macro\_arg\_guard pragma}语法中,\texttt{pragma}后面的标记是宏参数名,以便在下一个宏函数中检入。由于前面代码片段中的\texttt{val * 94}表达式中的\texttt{val}没有括在括号中,编译时将输出以下警告消息:

\begin{tcblisting}{commandshell={}}
$ clang … foo.c
[WARNING] In foo.c:3:18: macro argument 'val' is not enclosed by parenthesis
\end{tcblisting}

虽然只是一个简单的例子,但实际上在宏函数变得非常大或复杂时非常有用。这种情况下,手动为每个宏参数添加括号可能很容易出错。一个能够捕捉这种错误的工具绝对是有帮助的。

在我们深入编码部分之前,先建立项目文件夹。以下是文件夹结构:

\begin{tcolorbox}[colback=white,colframe=black]
\tt
\zihao{-5}
MacroGuard \\
\hspace*{0.5cm}|\_\_\_ CMakeLists.txt \\
\hspace*{0.5cm}|\_\_\_ MacroGuardPragma.cpp \\
\hspace*{0.5cm}|\_\_\_ MacroGuardValidator.h \\
\hspace*{0.5cm}|\_\_\_ MacroGuardValidator.cpp
\end{tcolorbox}

\texttt{MacroGuardPragama.cpp}文件包括一个自定义的\texttt{PragmaHandler}函数,我们将在下一节实现这个函数时再讨论。对于\texttt{MacroGuardValidator.h/.cpp},包括自定义的\texttt{PPCallbacks}函数,用于检查指定的宏体和参数是否符合这里的规则。

因为这个项目是一个LLVM源码树之外的项目,如果不知道如何导入LLVM自己的CMake指令(比如\texttt{add\_llvm\_library}和\texttt{add\_llvm\_executable}),请参考第2章中相关章节。在这里也要处理Clang,所以需要使用类似的方式来导入Clang的构建配置,例如如下代码所示的包含文件夹路径:

\begin{lstlisting}[style=styleCMake]
# In MacroGuard/CmakeLists.txt
…
# (after importing LLVM's CMake directives)
find_package(Clang REQUIRED CONFIG)
include_directories(${CLANG_INCLUDE_DIRS}
\end{lstlisting}

这里不需要设置Clang库路径的原因是:通常情况下,插件会动态链接到加载程序提供的库的实现(在我们的例子中,是\texttt{clang}可执行文件),而不是在构建时显式链接那些库。

最后,添加插件的构建目标,如下所示:

\begin{lstlisting}[style=styleCMake]
set(_SOURCE_FILES
	MacroGuardPragma.cpp
	MacroGuardValidator.cpp
	)
add_llvm_library(MacroGuardPlugin MODULE
	${_SOURCE_FILES}
	PLUGIN_TOOL clang)
\end{lstlisting}

\begin{tcolorbox}[colback=blue!5!white,colframe=blue!75!black, fonttitle=\bfseries,title=PLUGIN\_TOOL的参数]
\hspace*{0.7cm}在前面的代码片段中看到的\texttt{add\_llvm\_library}的\texttt{PLUGIN\_TOOL}参数实际上是专为Windows平台设计的,因为动态链接库(DLL)文件——Windows中的动态共享对象文件格式——有一个……有趣的规则,要求加载器可执行文件的名称显示在DLL文件头中。\texttt{PLUGIN\_TOOL}用于指定插件加载程序的名称。
\end{tcolorbox}

建立了CMake脚本并构建了插件之后,可以使用以下命令运行插件:

\begin{tcblisting}{commandshell={}}
$ clang … -fplugin=/path/to/MacroGuardPlugin.so foo.c
\end{tcblisting}

当然,目前还没有编写任何代码,所以什么也没有打印出来。下一节中,我们将开发一个自定义的\texttt{PragmaHandler}实例来实现我们新的\texttt{\#pragma macro\_arg\_guard}语法。

\subsubsubsection{6.4.2\hspace{0.2cm}实现自定义pragma}

实现上述特性的第一步是创建一个自定义的\texttt{\#pragma}处理程序。为此,首先创建一个\texttt{MacroGu\\ardHandler}类,其从\texttt{MacroGuardPragma.cpp}文件中的\texttt{PragmaHandler}类派生而来,如下所示:

\begin{lstlisting}[style=styleCXX]
struct MacroGuardHandler : public PragmaHandler {
	MacroGuardHandler() : PragmaHandler("macro_arg_guard"){}
	void HandlePragma(Preprocessor &PP, PragmaIntroducer
					          Introducer, Token &PragmaTok) override;
};
\end{lstlisting}

每当预处理程序遇到非标准的编译指令时,将调用\texttt{HandlePragma}回调函数。我们将在这个函数中做两件事,如下所示:

\begin{enumerate}
\item 检索\texttt{pragma name}标记(\texttt{macro\_arg\_guard})之后的任何补充标记(作为\texttt{pragma}参数处理)。

\item 注册一个\texttt{PPCallbacks}实例,扫描下一个宏函数定义的主体,看看特定的宏参数是否正确地用括号括起来。下面将概述这项任务的细节。
\end{enumerate}

对于第一个任务,我们利用\texttt{Preprocessor}来帮助我们解析pragma参数,这些参数是要封装的宏参数名。当调用\texttt{HandlePragma}时,\texttt{Preprocessor}则停在\texttt{pragma name}标记后面的位置,如下面的代码所示:

\begin{tcblisting}{commandshell={}}
#pragma macro_arg_guard val
                           ^--Stop at here
\end{tcblisting}

因此,我们需要做的就是继续词法分析和存储这些令牌,直到到达这一行的末尾:

\begin{lstlisting}[style=styleCXX]
void MacroGuardHandler::HandlePragma(Preprocessor &PP,…) {
	Token Tok;
	PP.Lex(Tok);
	while (Tok.isNot(tok::eod)) {
		ArgsToEnclosed.push_back(Tok.getIdentifierInfo());
		PP.Lex(Tok);
	}
}
\end{lstlisting}

上述代码片段中的\texttt{eod}令牌类型表示\textbf{指令的结束}。它专门用于标记预处理指令的结束。

对于\texttt{ArgsToEscped}变量,下面的全局数组存储了指定宏参数的\texttt{IdentifierInfo}对象:

\begin{lstlisting}[style=styleCXX]
SmallVector<const IdentifierInfo*, 2> ArgsToEnclosed;
struct MacroGuardHandler: public PragmaHandler {
	…
};
\end{lstlisting}

我们将\texttt{ArgsToEnclosed}声明在全局作用域中的原因是,稍后要使用它与我们的\texttt{PPCallbacks}实例通信,后者将使用该数组的内容来执行验证。

尽管我们的\texttt{PPCallbacks}实例,即\texttt{MacroGuardValidator}类的实现细节要在下一节才讨论,但当\texttt{HandlePragma}函数第一次调用时,它需要在预处理器中注册,如下所示:

\begin{lstlisting}[style=styleCXX]
struct MacroGuardHandler : public PragmaHandler {
	bool IsValidatorRegistered;
	MacroGuardHandler() : PragmaHandler("macro_arg_guard"),
	IsValidatorRegistered(false) {}
	…
};
void MacroGuardHandler::HandlePragma(Preprocessor &PP,…) {
	…
	if (!IsValidatorRegistered) {
		auto Validator = std::make_unique<MacroGuardValidator>(…);
		PP.addCallbackPPCallbacks(std::move(Validator));
		IsValidatorRegistered = true;
	}
}
\end{lstlisting}

我们还使用一个标志来确保它只注册一次。在此之后,无论何时发生预处理事件,都将调用我们的\texttt{MacroGuardValidato}r类来处理。我们的例子中,只对宏定义事件感兴趣,该事件向\texttt{MacroGuardValidator}发出信号,以验证刚刚定义的宏体。

在结束\texttt{PragmaHandler}之前,我们需要一些额外的代码将处理程序转换成一个插件,如下所示:

\begin{lstlisting}[style=styleCXX]
struct MacroGuardHandler : public PragmaHandler {
	…
};
static PragmaHandlerRegistry::Add<MacroGuardHandler>
  X("macro_arg_guard", "Verify if designated macro args are
    enclosed");
\end{lstlisting}

声明了这个变量之后,当这个插件加载到\texttt{clang}时,一个\texttt{MacroGuardHandler}实例将插入到全局的\texttt{PragmaHandler}注册表中,当它遇到一个非标准的\texttt{\#pragma}指令时,预处理器将查询注册表。当插件加载时,Clang就能够识别我们自定义的\texttt{macro\_arg\_guard}。

\subsubsubsection{6.4.3\hspace{0.2cm}实现自定义预处理器的回调}

\texttt{Preprocessor}提供了一组回调函数,即\texttt{PPCallbacks}类,当某些预处理器事件(比如:宏扩展)发生时,这些回调函数将触发。上一节,展示了如何注册自己的\texttt{PPCallbacks}实现,即\texttt{MacroGuardValidator}与\texttt{Preprocessor}。这里,展示了\texttt{MacroGuardValidator}如何在宏函数中验证宏参数-转义的规则。

首先,在\texttt{MacroGuardValidator.h/.cpp}中,放入以下框架:

\begin{lstlisting}[style=styleCXX]
// In MacroGuardValidator.h
extern SmallVector<const IdentifierInfo*, 2> ArgsToEnclosed;

class MacroGuardValidator : public PPCallbacks {
	SourceManager &SM;
public:
	explicit MacroGuardValidator(SourceManager &SM) : SM(SM) {}
	void MacroDefined(const Token &MacroNameToke,
	const MacroDirective *MD) override;
};

// In MacroGuardValidator.cpp
void MacroGuardValidator::MacroDefined(const Token
&MacroNameTok, const MacroDirective *MD) {
}
\end{lstlisting}

在\texttt{PPCallbacks}的所有回调函数中,我们只对\texttt{MacroDefined}感兴趣,它在宏定义处理时调用,由\texttt{MacroDirective}类型函数参数(\texttt{MD})表示。需要显示一些警告消息时,\texttt{SourceManager}类型成员变量(\texttt{SM})用于打印\texttt{SourceLocation}。

这里关注于\texttt{MacroGuardValidator::MacroDefined},其实逻辑非常简单:对于\texttt{ArgsToEnclosed}数组中的每个标识符,我们都扫描宏体,以检查它的出现是否有括号作为它的前代和后代令牌。首先,让我们将其放入循环的框架,如下所示:

\begin{lstlisting}[style=styleCXX]
void MacroGuardValidator::MacroDefined(const Token
&MacroNameTok, const MacroDirective *MD) {
	const MacroInfo *MI = MD->getMacroInfo();
	// For each argument to be checked…
	for (const IdentifierInfo *ArgII : ArgsToEnclosed) {
		// Scanning the macro body
		for (auto TokIdx = 0U, TokSize = MI->getNumTokens();
		TokIdx < TokSize; ++TokIdx) {
			…
		}
	}
}
\end{lstlisting}

如果一个宏体令牌的\texttt{IdentifierInfo}参数与\texttt{ArgII}匹配,这意味着出现了一个宏参数,我们检查该令牌的前一个和下一个令牌,如下所示:

\begin{lstlisting}[style=styleCXX]
for (const IdentifierInfo *ArgII : ArgsToEnclosed) {
	for (auto TokIdx = 0U, TokSize = MI->getNumTokens();
	TokIdx < TokSize; ++TokIdx) {
		Token CurTok = *(MI->tokens_begin() + TokIdx);
		if (CurTok.getIdentifierInfo() == ArgII) {
			if (TokIdx > 0 && TokIdx < TokSize - 1) {
				auto PrevTok = *(MI->tokens_begin() + TokIdx - 1),
				NextTok = *(MI->tokens_begin() + TokIdx + 1);
				if (PrevTok.is(tok::l_paren) && NextTok.is
				(tok::r_paren))
				continue;
			}
			…
		}
	}
}
\end{lstlisting}

\begin{tcolorbox}[colback=blue!5!white,colframe=blue!75!black, fonttitle=\bfseries,title=\texttt{IdentifierInfo}实例的唯一性]
\hspace*{0.7cm}回想一下,相同的标识符字符串总是由相同的\texttt{IdentifierInfo}对象表示。这就是这里可以简单地使用指针比较的原因。
\end{tcolorbox}

\texttt{MacroInfo::tokens\_begin}函数返回一个迭代器,指向一个包含所有宏体令牌的数组开头。

最后,如果宏参数令牌没有括起来,则会打印一个警告消息,如下所示:

\begin{lstlisting}[style=styleCXX]
for (const IdentifierInfo *ArgII : ArgsToEnclosed) {
	for (auto TokIdx = 0U, TokSize = MI->getNumTokens();
	TokIdx < TokSize; ++TokIdx) {
		…
		if (CurTok.getIdentifierInfo() == ArgII) {
			if (TokIdx > 0 && TokIdx < TokSize - 1) {
				…
				if (PrevTok.is(tok::l_paren) && NextTok.is
				(tok::r_paren))
				continue;
			}
			SourceLocation TokLoc = CurTok.getLocation();
			errs() << "[WARNING] In " << TokLoc.printToString(SM)
			<< ": ";
			errs() << "macro argument '" << ArgII->getName()
			<< "' is not enclosed by parenthesis\n";
		}
	}
}
\end{lstlisting}

这部分就讲到这里了。读者们现在可以开发一个\texttt{PragmaHandler}插件,可以动态加载到Clang中来处理自定义的\texttt{\#pragma}指令。还了解了如何实现\texttt{PPCallbacks},以便在预处理器事件发生时插入自定义逻辑。

















