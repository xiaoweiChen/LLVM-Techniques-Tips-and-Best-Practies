本章从预处理器和词法分析器标记前端的开始。前者用其他文本内容替换预处理器指令,而后者将源代码划分成更有意义的标记(或令牌)。本章中,我们学习了这两个组件如何相互协作,以提供一个令牌流的视图,以便在后面的阶段中使用。此外,还学习了各种重要的API,比如:\texttt{Preprocessor}类、\texttt{Token}类,以及如何在Clang中表示宏——这些API可以用于本部分的开发,特别是用于创建支持自定义\texttt{\#pragma}指令的处理程序插件,以及创建定制的预处理器回调,以便与预处理事件进行更深入的集成。

按照Clang编译阶段的顺序,下一章将展示如何使用\textbf{抽象语法树(AST)},以及如何开发一个AST插件,并将自定义逻辑插入其中。