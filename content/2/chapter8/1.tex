本章中,我们仍然依赖于\texttt{clang}可执行文件,请先构建它:

\begin{tcblisting}{commandshell={}}
$ ninja clang
\end{tcblisting}

因为使用了驱动,所以可以使用\texttt{-\#\#\#}命令行选项来打印从驱动程序中转译出来的前端标志:

\begin{tcblisting}{commandshell={}}
$ clang++ -### -std=c++11 -Wall hello_world.cpp -o hello_world
"/path/to/clang" "-cc1" "-triple" "x86_64-apple-macosx11.0.0"
"-Wdeprecated-objc-isa-usage" 
"-Werror=deprecated-objcisa-usage" "-Werror=implicit-function-declaration" 
"-emitobj" "-mrelax-all" "-disable-free" "-disable-llvm-verifier"
… "-fno-strict-return" "-masm-verbose" "-munwind-tables"
"-target-sdk-version=11.0" … "-resource-dir" "/Library/
Developer/CommandLineTools/usr/lib/clang/12.0.0" "-isysroot"
"/Library/Developer/CommandLineTools/SDKs/MacOSX.sdk" "-I/
usr/local/include" "-stdlib=libc++" … "-Wall" "-Wno-reorderinit-list" 
\end{tcblisting}
\begin{tcblisting}{commandshell={}}
"-Wno-implicit-int-float-conversion" "-Wno-c99-
designator" … "-std=c++11" "-fdeprecated-macro"
 "-fdebugcompilation-dir" 
"/Users/Rem" "-ferror-limit" "19"
"-fmessage-length" "87" "-stack-protector" "1" "-fstackcheck" 
"-mdarwin-stkchk-strong-link" … "-fexceptions" …
"-fdiagnostics-show-option" "-fcolor-diagnostics" "-o" "/path/
to/temp/hello_world-dEadBeEf.o" "-x" "c++" "hello_world.cpp"…
\end{tcblisting}

使用这个标志将不会运行其余的编译,而只是执行驱动程序和工具链。这个方法可以验证和调试特定标志,并检查它们是否正确地从驱动传播到前端。

最后,本章的最后一节,添加自定义工具链。我们将处理一个只能在Linux系统上运行的项目。另外,请提前安装OpenSSL。在大多数Linux系统中,它通常是一个包。例如,在Ubuntu上,可以使用以下命令来安装:

\begin{tcblisting}{commandshell={}}
$ sudo apt install openssl
\end{tcblisting}

我们只使用命令行工具,所以不需要安装任何通常用于开发的OpenSSL库。

本章的代码连接: \url{https://github.com/PacktPublishing/LLVM-Techniques-Tips-and-Best-Practices-Clangand-Middle-End-Libraries/tree/main/Chapter08}.

本章的第一节中,我们将简要介绍Clang的驱动和工具链基础结构。




















