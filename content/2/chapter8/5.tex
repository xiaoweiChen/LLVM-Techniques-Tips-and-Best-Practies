In this chapter, we started by introducing Clang's driver and the role of the toolchain – the module that provides platform-specific information such as the supported assemblers and linkers – that assisted it. Then, we showed you one of the most common ways to customize the driver – adding a new driver flag. After that, we talked about the toolchain and, most importantly, how to create a custom one. These skills are really useful when you want to create a new feature in Clang (or even LLVM) and need a custom compiler flag to enable it. Also, the ability to develop a custom toolchain is crucial for supporting Clang on new operating systems, or even new hardware architecture.

This is the final chapter of the second part of this book. Starting from the next chapter, we will talk about LLVM's middle end – the platform-independent program analysis and optimization framework.