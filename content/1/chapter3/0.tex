In the previous chapter, we learned how to take advantage of LLVM's own CMake utilities to improve our development experience. We also learned how to seamlessly integrate LLVM into other out-of-tree projects. In this chapter, we're going to talk about how to get hands-on with LLVM's own testing infrastructure, LIT.

LIT is a testing infrastructure that was originally developed for running LLVM's regression tests. Now, it's not only the harness for running all the tests in LLVM (both unit and regression tests) but also a generic testing framework that can be used outside of LLVM. It also provides a wide range of testing formats to tackle different scenarios. This chapter will give you a thorough tour of the components in this framework and help you master LIT.

本章中,我们将讨论以下主题:

\begin{itemize}
	\item Using LIT in out-of-tree projects
	\item Learning about advanced FileCheck tricks
	\item Exploring the TestSuite framework
\end{itemize}









































