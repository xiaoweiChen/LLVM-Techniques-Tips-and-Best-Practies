LIT is a general-purpose testing framework that can not only be used inside LLVM, but also arbitrary projects with little effort. This chapter tried to prove this point by showing you how to integrate LIT into an out-of-tree project without even needing to build LLVM. Second, we saw FileCheck – a powerful pattern checker that's used by many LIT test scripts. These skills can reinforce the expressiveness of your testing scripts. Finally, we presented you with the TestSuite framework, which is suitable for testing different kinds of program and complements the default LIT testing format.

In the next chapter, we will explore another supporting framework in the LLVM project: TableGen. We will show you that TableGen is also a general toolbox that can solve problems in out-of-tree projects, albeit almost being exclusively used by backend development in LLVM nowadays.