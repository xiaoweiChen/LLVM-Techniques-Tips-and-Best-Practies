Writing an in-tree LLVM IR regression test is pretty easy: all you need to do is annotate the IR file with testing directives. Look at the following script, for example:

\begin{tcolorbox}[colback=white,colframe=black]
\tt
; RUN: opt < \%s -instcombine -S -o - | FileCheck \%s \\
target triple = "x86\_64-unknown-linux" \\
define i32 @foo(i32 \%c) \{ \\
\hspace*{0.3cm}entry: \\
\hspace*{0.3cm}; CHECK: [[RET:\%.+]] = add nsw i32 \%c, 3 \\
\hspace*{0.3cm}; CHECK: ret i32 [[RET]] \\
\hspace*{0.3cm}\%add1 = add nsw i32 \%c, 1 \\
\hspace*{0.3cm}\%add2 = add nsw i32 \%add1, 2 \\
\hspace*{0.3cm}ret i32 \%add2 \\
\}
\end{tcolorbox}

This script checks if InstCombine (triggered by the -instcombine command-line option shown in the preceding snippet) simplifies two succeeding arithmetic adds into one. After putting this file into an arbitrary folder under llvm/test, the script will automatically be picked and run as part of the regression test when you're executing the llvm-lit command-line tool.

Despite its convenience, this barely helps you use LIT in out-of-tree projects. Using LIT out-of-tree is especially useful when your project needs some end-to-end testing facilities, such as a format converter, a text processor, a linter, and, of course, a compiler. This section will show you how to bring LIT to your out-of-tree projects, and then provide you with a complete picture of the running flow of LIT.

\subsubsubsection{3.2.1\hspace{0.2cm}Preparing for our example project}

In this section, we will use an out-of-tree CMake project. This example project builds a command-line tool, js-minifier, that minifies arbitrary JavaScript code. We will transform the following JavaScript code:

\begin{lstlisting}[style=styleCMake]
const foo = (a, b) => {
	let c = a + b;
	console.log(`This is ${c}`);
}
\end{lstlisting}

This will be transformed into some other semantic-equivalent code that is as short as possible:

\begin{lstlisting}[style=styleCMake]
const foo = (a,b) => {let c = a + b; console.log(`This is ${c}`);}
\end{lstlisting}

Instead of teaching you how to write this js-minifier, the goal of this section is to
show you how to create a LIT testing environment to test this tool.

The example project has the following folder structure:

\begin{tcolorbox}[colback=white,colframe=black]
\tt
/JSMinifier \\
\hspace*{0.5cm}|\_\_\_ CMakeLists.txt \\
\hspace*{0.5cm}|\_\_\_ /src \\
\hspace*{0.5cm}|\hspace{1cm}|\_\_\_ js-minifier.cpp \\
\hspace*{0.5cm}|\hspace{1cm}|\_\_\_ /test \\
\hspace*{0.5cm}|\hspace{1.5cm}|\_\_\_ test.js \\
\hspace*{0.5cm}|\hspace{1.5cm}|\_\_\_ CMakeLists.txt \\
\hspace*{0.5cm}|\_\_\_ /build
\end{tcolorbox}

The files under the /src folder contain the source code for js-minifier (which we are not going to cover here). What we will focus on here are the files that will be used for testing js-minifier, which sit under the /test folder (for now, there is only one file, test.js).

In this section, we are going to set up a testing environment so that when we run llvmlit – the testing driver and main character of this section – under the CMake /build folder, it will print testing results, like this:

\begin{tcblisting}{commandshell={}}
$ cd build
$ llvm-lit -sv .
-- Testing: 1 tests, 1 workers –
PASS: JSMinifier Test :: test.js (1 of 1)
Testing Time: 0.03s
  Expected Passes : 1
\end{tcblisting}

This shows how many and what test cases have passed.

Here is the testing script, test.js:

\begin{lstlisting}[style=styleCMake]
// RUN: %jsm %s -o - | FileCheck
// CHECK: const foo = (a,b) =>
// CHECK-SAME: {let c = a + b; console.log(`This is ${c}`);}
const foo = (a, b) => {
	let c = a + b;
	console.log(`This is ${c}`);
}
\end{lstlisting}

As you can see, it is a simple testing process that runs the js-minifier tool – represented by the \%jsm directive, which will be replaced by the real path to js-minifier executable, as explained later – and checks the running result with FileCheck by using its CHECK and CHECK-SAME directives.

With that, we've set up our example project. Before we wrap up the preparation, there is one final tool we need to create.

Since we're trying to cut down on our reliance on the LLVM source tree, recreate the llvm-lit command-line tool using the LIT package available in the PyPi repository (that is, the pip command-line tool). All you need to do is install that package:

\begin{tcblisting}{commandshell={}}
$ pip install --user lit
\end{tcblisting}

Finally, wrap the package with the following script:

\begin{lstlisting}[style=styleCMake]
#!/usr/bin/env python
from lit.main import main
if __name__ == '__main__':
	main()
\end{lstlisting}

Now, we can use LIT without building an LLVM tree! Next, we will create some LIT configuration scripts that will drive the whole testing flow.

\subsubsubsection{3.2.2\hspace{0.2cm}Writing LIT configurations}

In this subsection, we'll show you how to write LIT configuration scripts. These scripts describe the testing process – where the files will be tested, the testing environment (if we need to import any tool, for example), the policy when there is a failure, and so on. Learning these skills can greatly improve how you use LIT in places outside the LLVM tree. Let's get started:


\subsubsubsection{3.2.3\hspace{0.2cm}LIT internals}

















