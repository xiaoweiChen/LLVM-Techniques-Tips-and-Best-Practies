因为对构建系统灵活性的要求,LLVM的构建方式已经从GNU autoconf切换到CMake。从那以后,LLVM提出了许多定制的CMake函数、宏和规则来优化自己的使用。本节将为您概述其中最重要和最常用的几个。我们将学习如何以及何时使用它们。

\subsubsubsection{2.2.1\hspace{0.2cm}使用CMake添加新的库}

库是LLVM框架的基本构建块。要为一个新库编写CMakeLists.txt时,不要使用\texttt{add\_library}指令:

\begin{lstlisting}[style=styleCMake]
# In an in-tree CMakeLists.txt file…
add_library(MyLLVMPass SHARED
  MyPass.cpp) # Do NOT do this to add a new LLVM library
\end{lstlisting}

使用\texttt{add\_library}有几个缺点:

\begin{itemize}
\item 第1章在构建LLVM时,LLVM更喜欢使用全局CMake参数(即\texttt{BUILD\_SHARED\_LIBS})来控制所有的组件库是静态构建还是动态构建。使用内置指令很难做到这一点的。

\item 类似的,LLVM更喜欢使用一个全局的CMake参数来控制一些编译标志,比如是否启用\textbf{运行时类型信息(RTTI)}和代码库中的\textbf{\texttt{C++}异常处理}。

\item 通过使用定制的CMake函数/宏,LLVM可以创建自己的组件系统,这为开发人员提供了更高级别的抽象,可以以更容易的方式指定构建目标依赖项。

\end{itemize}

因此,可以使用\texttt{add\_llvm\_component\_library}:

\begin{lstlisting}[style=styleCMake]
# In a CMakeLists.txt
add_llvm_component_library(LLVMFancyOpt
  FancyOpt.cpp)
\end{lstlisting}

这里,\texttt{LLVMFancyOpt}是库名,\texttt{FancyOpt.cpp}是源文件。

常规的CMake脚本中,可以使用\texttt{target\_link\_libraries}来指定给定目标的库依赖关系,然后使用\texttt{add\_dependencies}在不同的构建目标之间,分配依赖关系来创建显式的描绘构建顺序。当使用LLVM的自定义CMake函数来创建库目标时,有一个更简单的方法来完成这些任务。

通过使用\texttt{add\_llvm\_component\_library}中的\texttt{LINK\_COMPONENTS}参数(或\texttt{add\_llvm\_library},是前一个的更底层实现),可以为目标指定需要链接的组件:

\begin{lstlisting}[style=styleCMake]
add_llvm_component_library(LLVMFancyOpt
	FancyOpt.cpp
	LINK_COMPONENTS
	Analysis ScalarOpts)
\end{lstlisting}

或者,可以对\texttt{LLVM\_LINK\_COMPONENTS}变量做同样的事情,而且需要在函数调用之前定义:

\begin{lstlisting}[style=styleCMake]
set(LLVM_LINK_COMPONENTS
	Analysis ScalarOpts)
add_llvm_component_library(LLVMFancyOpt
	FancyOpt.cpp)
\end{lstlisting}

当需要使用的LLVM构建块时,组件库不过是具有特殊意义的普通库。如果选择构建它,其就会包含在巨大的libLLVM库中,组件名与真正的库名会略有不同。如果需要从组件名映射到库名,可以使用下面的CMake函数:

\begin{lstlisting}[style=styleCMake]
add_llvm_component_library(LLVMFancyOpt
	FancyOpt.cpp
	LINK_LIBS
	${BOOST_LIBRARY})
\end{lstlisting}

如果想直接链接到一个普通的库(非LLVM组件),可以使用\texttt{LINK\_LIBS}参数:

\begin{lstlisting}[style=styleCMake]
add_llvm_component_library(LLVMFancyOpt
	FancyOpt.cpp
	LINK_LIBS
	${BOOST_LIBRARY})
\end{lstlisting}

要将常规构建目标依赖分配给库目标(相当于\texttt{add\_dependencies}),可以使用\texttt{DEPENDS}参数:

\begin{lstlisting}[style=styleCMake]
add_llvm_component_library(LLVMFancyOpt
	FancyOpt.cpp
	DEPENDS
	intrinsics_gen)
\end{lstlisting}

\texttt{intrinsics\_gen}是一个常见的目标,表示生成包含LLVM intrinsics的头文件。

\hspace*{\fill} \\
\noindent
\textbf{为每个文件夹添加一个构建目标}

许多LLVM定制的CMake函数都有涉及到源文件检测陷阱。假设有一个这样的目录结构:

\begin{tcolorbox}[colback=white,colframe=black]
\tt
\zihao{-5}
/FancyOpt \\
\hspace*{0.5cm}|\_\_\_ FancyOpt.cpp \\
\hspace*{0.5cm}|\_\_\_ AggressiveFancyOpt.cpp \\
\hspace*{0.5cm}|\_\_\_ CMakeLists.txt
\end{tcolorbox}

这里,有两个源文件:FancyOpt.cpp和AggressiveFancyOpt.cpp。FancyOpt.cpp是这种优化的基线版本,而AggressiveFancyOpt.cpp是一个(更激进的)实现版本。通常,希望将它们独立成库,以便用户选择。所以,可以这样写一个CMakeLists.txt文件:

\begin{lstlisting}[style=styleCMake]
# In /FancyOpt/CMakeLists.txt
add_llvm_component_library(LLVMFancyOpt
	FancyOpt.cpp)
add_llvm_component_library(LLVMAggressiveFancyOpt
	AggressiveFancyOpt.cpp)
\end{lstlisting}

不幸的是,当处理第一个\texttt{add\_llvm\_component\_library}语句时,会产生错误:\texttt{Found unknown source AggressiveFancyOpt.cpp …}

LLVM的构建系统会执行更严格的规则,以确保同一文件夹中的所有\texttt{C/C++}源文件都能添加到相同的库、可执行文件或插件中。为了解决这个问题,需要将两个文件拆分到单独的文件夹中,如下所示:

\begin{tcolorbox}[colback=white,colframe=black]
\tt
/FancyOpt \\
\hspace*{0.5cm}|\_\_\_ FancyOpt.cpp \\
\hspace*{0.5cm}|\_\_\_ CMakeLists.txt \\
\hspace*{0.5cm}|\_\_\_ /AggressiveFancyOpt \\
\hspace*{1cm}|\_\_\_ AggressiveFancyOpt.cpp \\
\hspace*{1cm}|\_\_\_ CMakeLists.txt
\end{tcolorbox}

/FancyOpt/CMakeLists.txt:

\begin{lstlisting}[style=styleCMake]
add_llvm_component_library(LLVMFancyOpt
	FancyOpt.cpp)
add_subdirectory(AggressiveFancyOpt)
\end{lstlisting}

/FancyOpt/AggressiveFancyOpt/CMakeLists.txt:

\begin{lstlisting}[style=styleCMake]
add_llvm_component_library(LLVMAggressiveFancyOpt
	AggressiveFancyOpt.cpp)
\end{lstlisting}

这些是使用LLVM的自定义CMake指令为(组件)库添加构建目标的重要方式。接下来的两节中,将展示如何使用不同的(LLVM的)CMake指令来添加可执行文件和Pass插件构建目标。

\subsubsubsection{2.2.2\hspace{0.2cm}使用CMake函数添加可执行文件和工具}

与\texttt{add\_llvm\_component\_library}类似,添加一个新的可执行目标,可以使用\texttt{add\_llvm\_executable}或\texttt{add\_llvm\_tool}:

\begin{lstlisting}[style=styleCMake]
add_llvm_tool(myLittleTool
	MyLittleTool.cpp)
\end{lstlisting}

这两个函数语法相同,只有通过\texttt{add\_llvm\_tool}创建的目标将包含在安装过程中。还有一个全局的CMake变量\texttt{LLVM\_BUILD\_TOOLS},用来启用/禁用LLVM工具。

这两个函数也可以通过\texttt{DEPENDS}参数来设置依赖项,类似于前面的\texttt{add\_llvm\_library}。这里,只能使用\texttt{LLVM\_LINK\_COMPONENTS}来指定链接组件。

\subsubsubsection{2.2.3\hspace{0.2cm}使用CMake函数添加Pass插件}

虽然本书的后面章节会讨论Pass插件的开发,但了解如何为Pass插件添加构建目标现最为合适(早期的LLVM版本仍然使用\texttt{add\_llvm\_library}和一些特定的参数):

\begin{lstlisting}[style=styleCMake]
add_llvm_pass_plugin(MyPass
	HelloWorldPass.cpp)
\end{lstlisting}

\texttt{LINK\_COMPONENTS}、\texttt{LINK\_LIBS}和\texttt{DEPENDS}参数也可以在这里使用,其用法和功能与\\\texttt{add\_llvm\_component\_library}相同。

这是一些常见和重要的(与llvm相关的)CMake指令,使用这些指令不仅可以使CMake代码更简洁。如果想做一些源码树内的开发,还可以与LLVM的构建系统同步。下一节中,利用本章所了解到的知识,了解如何将LLVM集成到源码树外的CMake项目中。

\begin{tcolorbox}[colback=blue!5!white,colframe=blue!75!black, fonttitle=\bfseries,title=LLVM源码内/外的开发]
\hspace*{0.7cm}本书中,源码树内开发是直接向LLVM项目贡献代码,例如:修复LLVM Cug或向现有的LLVM库添加新特性。另一方面,源码树外开发,要么为LLVM创建扩展(例如,编写一个LLVM pass),要么在其他项目中使用LLVM库(例如,使用LLVM的代码生成库来实现属于自己的编程语言)。
\end{tcolorbox}














