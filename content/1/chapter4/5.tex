In this chapter, we introduced TableGen, a powerful DSL for expressing structural data. We have shown you its universality in solving a variety of tasks, albeit it originally being created for compiler development. Through the lens of writing a donut recipe in TableGen, we have learned its core syntax. The following section on developing a custom TableGen backend taught you how to use C++ APIs to interact with in-memory TableGen directives parsed from the source input, giving you the power to create a complete and standalone TableGen toolchain to implement your own custom logic. Learning how to master TableGen can not only help your development in LLVM-related projects but also gives you more options to solve structural data problems in arbitrary projects.

This section marks the end of the first part—an introduction to all kinds of useful supporting components in the LLVM project. Starting from the next chapter, we will move into the core compilation pipeline of LLVM. The first important topic we will cover is Clang, LLVM's official frontend for C-family programming languages.