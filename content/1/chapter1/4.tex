CMake的可移植和灵活性非常好,已经经过了许多工业项目的实战测试。然而,当涉及到重新配置时,它就会有一些严重的问题。正如我们在前几节中看到的,已生成的构建文件,可以通过编辑build文件夹中的CMakeCache.txt文件来修改一些CMake参数。当你再次调用构建命令时,CMake会重新配置构建文件。如果在源文件夹中编辑CMakeLists.txt文件,同样的重新配置也会出现。CMake的重配置过程主要有两个缺点:

\begin{itemize}
\item 在某些系统中,CMake的配置过程非常缓慢。即使是理论上只运行部分流程的重构,有时仍然需要很长时间。

\item 有时CMake会无法解决不同变量和构建目标之间的依赖关系,所以你的更改不会反映出这一点。最糟糕的情况是,它会悄无声息地失败,并花费您很长时间来查找问题。
	
\end{itemize}

Ninja,更广为人知的名字是GN,是谷歌的许多项目使用的一个构建文件生成器,比如Chromium。GN从它自己的描述语言生成Ninja文件。它具有快速配置时间和可靠的参数管理的良好声誉。LLVM自2018年末(大约版本8.0.0)以来,已经将GN支持作为一种可选的(实验性的)构建方法。如果您的开发对构建文件进行了更改,或者您想在短时间内尝试不同的构建选项,那么GN尤其有用。

使用GN构建LLVM的步骤如下:

\begin{enumerate}
\item LLVM的GN支持位于llvm/utils/gn文件夹中。切换到该文件夹后,运行以下get.py脚本,在本地下载GN的可执行文件:

\begin{tcblisting}{commandshell={}}
$ cd llvm/utils/gn
$ ./get.py
\end{tcblisting}

\begin{tcolorbox}[colback=blue!5!white,colframe=blue!75!black, fonttitle=\bfseries,title=使用特定版本的GN]
\hspace*{0.7cm}如果希望使用自定义GN可执行文件,而不是get.py获取的可执行文件,只需将特定版本的GN放入系统的PATH中。如果您想知道还有哪些其他GN版本可用,您可以在\url{https://dev.chromium.org/developers/how-tos/install-depot-tools}查看关于安装depot\_tools的信息。
\end{tcolorbox}

\item 在同一个文件夹中使用gn.py生成构建文件(本地版本的gn.py只是一个包装器,用于设置基本环境):

\begin{tcblisting}{commandshell={}}
$ ./gn.py gen out/x64.release
\end{tcblisting}

out/x64.release是构建文件夹的名称。通常,GN用户的文件命名规则为\texttt{<architecture>.<build type>.<other features>}。

\item 最后,可以切换到构建文件夹并启动Ninja:

\begin{tcblisting}{commandshell={}}
$ cd out/x64.release
$ ninja <build target>
\end{tcblisting}

\item 或者,使用\texttt{-C}选项:

\begin{tcblisting}{commandshell={}}
$ ninja -C out/x64.release <build target>
\end{tcblisting}
	
\end{enumerate}

您可能已经知道初始构建文件生成过程非常快。现在,如果您想更改一些构建参数,请找到args.gn文件,在build文件夹下(out/x64.release/args.gn)。如果想改变构建类型来调试和改变目标来构建(修改\texttt{LLVM\_TARGETS\_TO\_BUILD} CMake参数)到X86和AArch64。建议使用以下命令启动一个编辑器来编辑args.gn:

\begin{tcblisting}{commandshell={}}
$ ./gn.py args out/x64.release
\end{tcblisting}

在args.gn中,输入如下内容:

\begin{tcolorbox}[colback=white,colframe=black]
\tt
\zihao{-5}
\# Inside args.gn \\
is\_debug = true \\
llvm\_targets\_to\_build = ["X86", "AArch64"]
\end{tcolorbox}

保存并退出编辑器后,GN将执行一些语法检查并重新生成构建文件(当然,您可以不使用gn命令编辑args.gn,这样在调用ninja命令之前,构建文件不会重新生成),这种重新生成/重新配置也会很快。最重要的是,不会有任何不确定的行为。由于GN的语言设计,可以很容易地分析不同构建参数之间的关系,几乎没有歧义。

GN的构建参数列表可以通过以下命令找到:

\begin{tcblisting}{commandshell={}}
$ ./gn.py args --list out/x64.release
\end{tcblisting}

不幸的是,在写这本书的时候,仍然有很多CMake参数没有移植到GN中。GN不是LLVM现有的CMake构建系统的替代品,但它是一个替代方案。尽管如此,如果希望在涉及许多构建配置更改的开发中获得快速的处理时间,那么GN仍然是一个不错的构建方法。






















