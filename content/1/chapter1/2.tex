As we mentioned at the beginning of this chapter, if you build LLVM with the default (CMake) configurations, by invoking CMake and building the project in the following way, there is a high chance that the whole process will take hours to finish:

\begin{tcblisting}{commandshell={}}
$ cmake ../llvm
$ make all
\end{tcblisting}

This can be avoided by simply using better tools and changing some environments. In this section, we will cover some guidelines to help you choose the right tools and configurations that can both speed up your building time and improve memory footprints

\subsubsubsection{Replacing GNU Make with Ninja}

The first improvement we can do is using the Ninja build tool (https://ninja-build.org) rather than GNU Make, which is the default build system generated by CMake on major Linux/Unix platforms.





























