At the time of writing this book, LLVM only has a few software requirements:

\begin{itemize}
	\item A C/C++ compiler that supports C++14
	\item CMake
	\item One of the build systems supported by CMake, such as GNU Make or Ninja
	\item Python (2.7 is fine too, but I strongly recommend using 3.x)
	\item zlib
\end{itemize}

The exact versions of these items change from time to time. Check out \url{https://llvm.org/docs/GettingStarted.html#software} for more details.

This chapter assumes you have built an LLVM before. If that's not the case, perform the following steps:

\begin{enumerate}
	\item Grab a copy of the LLVM source tree from GitHub:
\begin{tcblisting}{commandshell={}}
$ git clone https://github.com/llvm/llvm-project
\end{tcblisting}

	\item Usually, the default branch should build without errors. If you want to use release versions that are more stable, such as release version 10.x, use the following command:
\begin{tcblisting}{commandshell={}}
$ git clone -b release/10.x \
    https://github.com/llvm/llvmproject
\end{tcblisting}

	\item Finally, you should create a build folder where you're going to invoke the CMake command. All the building artifacts will also be placed inside this folder. This can be done using the following command:
\begin{tcblisting}{commandshell={}}
$ mkdir .my_build
$ cd .my_build
\end{tcblisting}
\end{enumerate}





